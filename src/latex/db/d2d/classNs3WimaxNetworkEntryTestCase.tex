version https://git-lfs.github.com/spec/v1
oid sha256:8064f2302f2661b6a881a13b1fe3c0494924fe0ad8e22cedd4d49aeb87c9ff49
size 6862
